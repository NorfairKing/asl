\documentclass[11pt]{article}
\usepackage[a4paper, portrait, margin=1in]{geometry}

\newcommand{\myname}{Tom Sydney Kerckhove}
\newcommand{\mynetzh}{tomk}
\newcommand{\myleginr}{15-908-064}

\newcommand{\javafile}[1]{\footnote{\url{https://gitlab.inf.ethz.ch/\mynetzh/asl-fall16-project/blob/master/asl/src/ch/ethz/asl/#1.java}}}
\newcommand{\plot}[1]{plots/#1}
\newcommand{\asset}[1]{assets/#1}


\newcommand{\java}[1]{\mintinline{java}{#1}}


\begin{document}

\title{Advanced Systems Lab (Fall'16) -- Third Milestone}

\author{Name: \emph{\myname}\\Legi number: \emph{\myleginr}}

\date{
\vspace{4cm}
\textbf{Grading} \\
\begin{tabular}{|c|c|}
\hline  \textbf{Section} & \textbf{Points} \\ 
\hline  1 &  \\ 
\hline  2 &  \\ 
\hline  3 &  \\ 
\hline  4 &  \\ 
\hline  5 &  \\ 
\hline \hline Total & \\
\hline 
\end{tabular} 
}

\maketitle
\newpage

\section{System as One Unit}\label{sec:system-one-unit}

% Length: 1-2 pages

% Build an M/M/1 model of your entire system based on the stability trace that you had to run for the first milestone. Explain the characteristics and behavior of the model built, and compare it with the experimental data (collected both outside and inside the middleware). Analyze the modeled and real-life behavior of the system (explain the similarities, the differences, and map them to aspects of the design or the experiments). Make sure to follow the model-related guidelines described in the Notes!

\section{Analysis of System Based on Scalability Data}\label{sec:analysis-scalability}

% Length: 1-4 pages

% Starting from the different configurations that you used in the second milestone, build M/M/m queuing models of the system as a whole. Detail the characteristics of these series of models and compare them with experimental data. The goal is the analysis of the model and the real scalability of the system (explain the similarities, the differences, and map them to aspects of the design or the experiments). Make sure to follow the model-related guidelines described in the Notes!

\begin{figure}
  \centering
  \input{\model{remote-stability-trace-mm1.tex}}
\end{figure}

\section{System as Network of Queues}\label{sec:network-of-queues}

% Length: 1-3 pages

% Based on the outcome of the different modeling efforts from the previous sections, build a comprehensive network of queues model for the whole system. Compare it with experimental data and use the methods discussed in the lecture and the book to provide an in-depth analysis of the behavior. This includes the identification and analysis of bottlenecks in your system. Make sure to follow the model-related guidelines described in the Notes!

\section{Factorial Experiment}\label{sec:2k-experiment}

% Length: 1-3 pages

% Design a $2^k$ factorial experiment and follow the best practices outlined in the book and in the lecture to analyze the results. You are free to choose the parameters for the experiment and in case you have already collected data in the second milestone that can be used as source for this experiment, you can reuse it. Otherwise, in case you need to run new experiments anyway, we recommend exploring the impact of request size on the middleware together with an other parameter.

\section{Interactive Law Verification}\label{sec:interactive-law}

% Length: 1-2 pages

% Check the validity of all experiments from one of the three sections in your Milestone 2 report using the interactive law (choose a section in which your system has at least 9 different configurations). Analyze the results and explain them in detail.

\end{document}
