\documentclass[11pt]{article}
\usepackage[a4paper, portrait, margin=1in]{geometry}

\newcommand{\myname}{Tom Sydney Kerckhove}
\newcommand{\mynetzh}{tomk}
\newcommand{\myleginr}{15-908-064}

\newcommand{\javafile}[1]{\footnote{\url{https://gitlab.inf.ethz.ch/\mynetzh/asl-fall16-project/blob/master/asl/src/ch/ethz/asl/#1.java}}}
\newcommand{\plot}[1]{plots/#1}
\newcommand{\asset}[1]{assets/#1}


\newcommand{\java}[1]{\mintinline{java}{#1}}


\usepackage{float}
\usepackage{hyperref}
\usepackage{tikz}
\usetikzlibrary{shapes}
\usetikzlibrary{patterns}

\begin{document}

\title{Advanced Systems Lab (Fall'16) -- Third Milestone}

\author{Name: \emph{\myname}\\Legi number: \emph{\myleginr}}

\date{
\vspace{4cm}
\textbf{Grading} \\
\begin{tabular}{|c|c|}
\hline  \textbf{Section} & \textbf{Points} \\ 
\hline  1 &  \\ 
\hline  2 &  \\ 
\hline  3 &  \\ 
\hline  4 &  \\ 
\hline  5 &  \\ 
\hline \hline Total & \\
\hline 
\end{tabular} 
}

\maketitle
\newpage

\section{Overview}

This report is the third installment in a series of three reports for the Advanced System Lab course.
The two previous reports considered the building of a middleware and experimentation.
This report will focus on modelling the middleware.

A model of the situation can be found in figure \ref{ref:complete-system}.

\begin{figure}[H]
  \centering
	
  \begin{tikzpicture}
    \node[anchor=south west,inner sep=0] (image) at (0,0) {\includegraphics[width=0.5\textwidth]{\asset{architecture.png}}};
    \begin{scope}[x={(image.south east)},y={(image.north west)}]
			\draw (-0.7,0.5) circle (1cm) node (C) {Clients};
			\node (N) at (-0.3,0.5) [cloud, draw,cloud puffs=10,cloud puff arc=120, aspect=2, inner ysep=1em] {network};
			\draw [->, thick] (C) -- (N) -- (image);
    \end{scope}
  \end{tikzpicture}
  \caption{Complete view of the system}
  \label{ref:complete-system}
\end{figure}

\section{System as One Unit}\label{sec:system-one-unit}

% Length: 1-2 pages

% Build an M/M/1 model of your entire system based on the stability trace that you had to run for the first milestone.
% Explain the characteristics and behavior of the model built, and compare it with the experimental data (collected both outside and inside the middleware).
% Analyze the modeled and real-life behavior of the system (explain the similarities, the differences, and map them to aspects of the design or the experiments).
% Make sure to follow the model-related guidelines described in the Notes!

\subsection{Boundaries of M/M/1 model}

To make a M/M/1 model of the entire system, we have to treat the system under test as a black box.
The black box is drawn from the middleware, across the network on the network on the server side and across the servers.

This black box seemed the most appropriate.
Adding the network on the client-side to the black box would add more network-traveling to the service that we are modelling.
On the other hand, drawing the black box around the middleware only makes it hard to define what the service of the model is.

In figure \ref{fig:mm1-black-box}, there is an illustration of the black box used for this model.

\begin{figure}[H]
  \centering
  \begin{tikzpicture}
    \node[anchor=south west,inner sep=0, opacity=0.4] (image) at (0,0) {\includegraphics[width=0.5\textwidth]{\asset{architecture.png}}};
    \begin{scope}[x={(image.south east)},y={(image.north west)}]
			\draw (-0.7,0.5) circle (1cm) node (C) {Clients};
			\node (N) at (-0.3,0.5) [cloud, draw,cloud puffs=10,cloud puff arc=120, aspect=2, inner ysep=1em] {network};
			\draw [->, thick] (C) -- (N) -- (image);
			\draw [pattern=north west lines, pattern color=blue, fill=black, fill opacity=0.3, text opacity=1] (0,0) rectangle (1.05,1.05);
    	\node[anchor=south west,inner sep=0] (image) at (0,0.31) {\includegraphics[width=0.5\textwidth]{\asset{mm1.png}}};
    \end{scope}
  \end{tikzpicture}
  \caption{Black box system for M/M/1 model}
  \label{fig:mm1-black-box}
\end{figure}


\subsection{Parameter estimation}

Next we pretend that everything in this black box corresponds to a system as modeled by an M/M/1 model, and try to estimate the parameters $\lambda$ and $\mu$ of the M/M/1 model.
To justify the following estimation of the parameters, we rely on the fact that we are dealing with a closed system.

The model parameters will be estimated on a stability trace experiment that was run in the first part of the project.
However, due to updates to the system, this experiment has had to be re-run.
The details of the configuration can be found in figure \ref{fig:sut-stability-trace}.

\begin{figure}[H]
  \centering
  \input{\genfile{remote-stability-trace-table.tex}}
  \caption{Stability trace}
  \label{fig:sut-stability-trace}
\end{figure}

\subsubsection{Estimation of the arrival rate $\lambda$}

The arrival rate $\lambda$ is estimated by the average\footnote{over the seconds of the trace} throughput as seen by the clients.
Indeed, because we are dealing with a closed system, the rate at which requests arrive at the black box is exactly equal to the rate at which requests get processed.

\subsubsection{Estimation of the service rate $\mu$}

The service rate $\mu$ is estimated by the maximum throughput throughput as seen by the clients over all the 1 second intervals of the experiment.
This can be justified by realising that the service rate is the maximum speed at which the system can process requests.
This means that taking the maximum throughput will yield a lower bound for the service rate.
Because we consider many one second intervals, this estimation should give us a relatively accurate lower-bound for the service rate.

\subsection{Comparison with experimental results}

In figure \ref{fig:mm1-model}, we can see the M/M/1 model that was built according to guidelines of the previous subsection.
On the left-hand side we can see the input parameters and output measures as predicted by the model.
On the right-hand side we can see a comparison with data gathered from the system.

\begin{figure}[H]
  \centering
  \input{\model{remote-stability-trace-mm1.tex}}
  \caption{M/M/1 model}
  \label{fig:mm1-model}
\end{figure}

It is important to note that 'response time' means the total time before the black box replies.
It has little to do with what the clients call the 'response time'.
The mean waiting time on the left means the time that a request spends in the M/M/1 model queue before being processed.

The data that these predictions are compared with are the best available data for the comparison:
The mean response time is compared to the difference between the average total time spent between receiving a request and sending a response by the middleware.
The mean waiting time is compared to the time between being enqueued and being dequeued by a request.

\subsection{Mapping between comparison and system components}

The M/M/1 model is a very limited model that cannot accurately represent the black box that it is covering.

\section{Analysis of System Based on Scalability Data}\label{sec:analysis-scalability}

% Length: 1-4 pages

% Starting from the different configurations that you used in the second milestone, build M/M/m queuing models of the system as a whole. Detail the characteristics of these series of models and compare them with experimental data. The goal is the analysis of the model and the real scalability of the system (explain the similarities, the differences, and map them to aspects of the design or the experiments). Make sure to follow the model-related guidelines described in the Notes!

\section{System as Network of Queues}\label{sec:network-of-queues}

% Length: 1-3 pages

% Based on the outcome of the different modeling efforts from the previous sections, build a comprehensive network of queues model for the whole system. Compare it with experimental data and use the methods discussed in the lecture and the book to provide an in-depth analysis of the behavior. This includes the identification and analysis of bottlenecks in your system. Make sure to follow the model-related guidelines described in the Notes!

\section{Factorial Experiment}\label{sec:2k-experiment}

% Length: 1-3 pages

% Design a $2^k$ factorial experiment and follow the best practices outlined in the book and in the lecture to analyze the results. You are free to choose the parameters for the experiment and in case you have already collected data in the second milestone that can be used as source for this experiment, you can reuse it. Otherwise, in case you need to run new experiments anyway, we recommend exploring the impact of request size on the middleware together with an other parameter.



\begin{figure}[H]
  \centering
  \input{\genfile{small-local-2k-factorial-sign-table.tex}}
  \caption{Sign table}
  \label{fig:sign-table}
\end{figure}

\section{Interactive Law Verification}\label{sec:interactive-law}

% Length: 1-2 pages

% Check the validity of all experiments from one of the three sections in your Milestone 2 report using the interactive law (choose a section in which your system has at least 9 different configurations). Analyze the results and explain them in detail.

We assume that the think time is negligible.
TODO: think time benchmark: 1 middleware, client and server, log every request, then think time is time between end of previous and start of next request.

TODO: Correct for the tail distribution?

\begin{figure}[H]
  \centering
  \input{\genfile{remote-replication-effect-irtl-table.tex}}
\end{figure}

\begin{figure}[H]
  \centering
  \includegraphics[width=\textwidth]{\plot{total-duration-histos/remote-replication-effect-histogram-5.png}}
\end{figure}

\begin{figure}[H]
  \centering
  \input{\genfile{remote-think-time-table.tex}}
  \caption{Think time benchmark}
  \label{fig:sut-think-time}
\end{figure}


\section{Log file listing}

\input{\genfile{report-3-logfile-listings.tex}}

\end{document}
