\documentclass[11pt]{article}
\usepackage[a4paper, portrait, margin=1in]{geometry}

\usepackage{mathtools}
\usepackage{float}
\usepackage{hyperref}
\usepackage{subcaption}

\newcommand{\myname}{Tom Sydney Kerckhove}
\newcommand{\mynetzh}{tomk}
\newcommand{\myleginr}{15-908-064}

\newcommand{\javafile}[1]{\footnote{\url{https://gitlab.inf.ethz.ch/\mynetzh/asl-fall16-project/blob/master/asl/src/ch/ethz/asl/#1.java}}}
\newcommand{\plot}[1]{plots/#1}
\newcommand{\asset}[1]{assets/#1}


\newcommand{\java}[1]{\mintinline{java}{#1}}


\begin{document}

\title{Advanced Systems Lab (Fall'16) -- Second
Milestone}

\author{Name: \emph{\myname}\\Legi number: \emph{\myleginr}}

\date{
\vspace{4cm}
\textbf{Grading} \\
\begin{tabular}{|c|c|}
\hline  \textbf{Section} & \textbf{Points} \\ 
\hline  1 &  \\ 
\hline  2 &  \\ 
\hline  3 &  \\ 
\hline \hline Total & \\
\hline 
\end{tabular} 
}

\maketitle

\newpage

\section{Introduction}

This is the second report in a series of three reports for the the advanced systems lab.
In this course we implement and investigate the performance of a middleware for Memcache.
This part focusses mostly on investigating the performance of the system that was built for the previous part.


\section{Fixes and upgrades}

The middleware has changed since the last part to improve its efficiency.
It no longer constructs the extremely verbose logging messages that it used to.
Even though these messages were never printed, they were still constructed because of Java's strict evaluation semantics.
This caused the middleware to perform very poorly.

\TODO{re-run stability trace?}

\section{Maximum Throughput}

\subsection{System under test}

% Find the highest throughput of your system for 5 servers with no replication and a read-only workload configuration.

In the first experiment, the middleware is in charge of five servers with no replication and a read-only workload configuration.
The aim is to investigate which combination of the total number of clients and the number of threads in each server's thread pool for reads achieves the highest throughput.
The details of the explored configuration can be found in figure \ref{fig:sut-maximum-throughput}.

\begin{figure}[H]
  \centering
  \input{\genfile{remote-maximum-throughput-table.tex}}
  \caption{Maximum throughput experiments}
  \label{fig:sut-maximum-throughput}
\end{figure}

\subsection{Hypothesis}

% Explain why the system reaches its maximum throughput at these points and show how the performance changes around these configurations.

Making an accurate estimate of the configuration that achieves the highest throughput is hard if not unnecessary.
What is more important is to estimate what will happen as we increase the total number of virtual clients or the number of threads in each server's thread pool for reads.

\subsubsection{Increasing the total number of virtual clients}

As the middleware operates with an unbounded queue of requests to process, it never drops requests even if the load increases.
\footnote{
  We assume that the limits of memory are not reached.
  This assumption is reasonable as the middleware machine that will be used has 14 GiB of main memory.
}

As the total number of virtual clients increases, we expect to see throughput increase linearly up to the point where request queues are no longer always empty.
From then on we expect to see throughput increase sublinearly up to the point where the request queues are no longer ever empty.
From then on the throughput should not increase any further as subsequents will just be punt onto the queue and requests will not be processed any quicker.

\subsubsection{Increasing the number of threads in each server's thread pool for reads}

Reads are executed in a synchronous manner.
This means that each read thread can perform at most one request per round-trip time.
Using more threads to processes read requests should somewhat solve this problem, but context switches are relatively expensive.
This means that we expect using multiple threads to increase the throughput up to a certain point.
At that point the cost of using multiple threads will outweigh the benefit.

In general we expect response time to remain constant up to the point where queues are no longer constantly empty.
After this point, any more work gets put onto the queue and we then expect response times to increase linearly.
\footnote{
  This is only true because the system under test is a closed system.
  The same argument would not necessarily hold for open systems.
}

\subsection{Results and analysis}

In figures \ref{fig:maximum-throughput-low}, \ref{fig:maximum-throughput-mid} and \ref{fig:maximum-throughput-high}, the results of the maximum throughput experiment are plotted.

In what follows, we will refer to the number of threads in the middleware's read request thread pools as the 'threadcount' of the setup.
For each threadcount under test there are four plots.

The first plot shows the throughput in function of the total number of virtual clients and the second shows the total response time.
Both of these first plots are derived from the data as viewed by the clients.

The next two plots show data gathered in the middleware.
They show a detailed breakdown of the time that an average request spends in the middleware.

\subsubsection{Maximum throughput configuration}

% What is the minimum number of threads and clients (rounded to multiple of 10) that together achieve this throughput?

An exact number for either the maximum throughput or the number of virtual clients at which this throughput is achieved is hard to compute.
Uncertainties in measurement, along with a vague definition of 'maximum throughput' make it hard to give an exact number.

The estimated maximum throughput and the number of virtual clients at which this throughput is achieved can be found in figure \ref{fig:maximum-throughput-numbers}.
The maximum throughput estimate is given with an error of plus-minus $500$ TPS.
The estimate for the number of virtual clients is given with an error of plus-minus $10$.

\begin{figure}[H]
  \centering
  \begin{tabular}{c|c|c}
    Number of threads & Maximum throughput (TPS) & Number of virtual clients \\
    \hline $1$ & $2000 \pm 500$ & $25$ \\
    \hline $5$ & $5000 \pm 500$ & $30$ \\
    \hline $9$ & $10000 \pm 500$ & $50$ \\
    \hline $13$ & $11000 \pm 500$ & $60$ \\
    \hline $17$ & $11500 \pm 500$ & $70$ \\
    \hline $21$ & $17500 \pm 500$ & $80$ \\
  \end{tabular}
  \label{fig:maximum-throughput-numbers}
\end{figure}

\begin{figure}[H]
  \centering
    \begin{subfigure}{0.48\textwidth}
    \includegraphics[width=\textwidth]{\plot{remote-maximum-throughput-maximum-throughput-1-tps.png}}
    \includegraphics[width=\textwidth]{\plot{remote-maximum-throughput-maximum-throughput-1-resp.png}}
    \end{subfigure}
    \begin{subfigure}{0.48\textwidth}
    \includegraphics[width=\textwidth]{\plot{remote-maximum-throughput-maximum-throughput-5-tps.png}}
    \includegraphics[width=\textwidth]{\plot{remote-maximum-throughput-maximum-throughput-5-resp.png}}
    \end{subfigure}
  \caption{Maximum Throughput}
  \label{fig:maximum-throughput-low}
\end{figure}
\begin{figure}[H]
  \centering
    \begin{subfigure}{0.48\textwidth}
    \includegraphics[width=\textwidth]{\plot{remote-maximum-throughput-maximum-throughput-9-tps.png}}
    \includegraphics[width=\textwidth]{\plot{remote-maximum-throughput-maximum-throughput-9-resp.png}}
    \end{subfigure}
    \begin{subfigure}{0.48\textwidth}
    \includegraphics[width=\textwidth]{\plot{remote-maximum-throughput-maximum-throughput-13-tps.png}}
    \includegraphics[width=\textwidth]{\plot{remote-maximum-throughput-maximum-throughput-13-resp.png}}
    \end{subfigure}
  \caption{Maximum Throughput}
  \label{fig:maximum-throughput-mid}
\end{figure}
\begin{figure}[H]
  \centering
    \begin{subfigure}{0.48\textwidth}
    \includegraphics[width=\textwidth]{\plot{remote-maximum-throughput-maximum-throughput-17-tps.png}}
    \includegraphics[width=\textwidth]{\plot{remote-maximum-throughput-maximum-throughput-17-resp.png}}
    \end{subfigure}
    \begin{subfigure}{0.48\textwidth}
    \includegraphics[width=\textwidth]{\plot{remote-maximum-throughput-maximum-throughput-21-tps.png}}
    \includegraphics[width=\textwidth]{\plot{remote-maximum-throughput-maximum-throughput-21-resp.png}}
    \end{subfigure}
  \caption{Maximum Throughput}
  \label{fig:maximum-throughput-high}
\end{figure}

\subsubsection{Comparison of results with hypothesis}

In each throughput plot we see that throughput first increases linearly, then increases sublinearly and finally levels off.
This corresponds exactly to the hypothesis.

In each response time plot, we see a linear increase in response time.
There is no phase of constant response time to see in these plots as hypothesized.
It is possible that the constant response time phase was too short to show up on the plots.

\subsubsection{Breakdown of time spent in middleware}

% Provide a detailed breakdown of the time spent in the middleware for each operation type.

Figures \ref{fig:maximum-throughput-slice-low}, \ref{fig:maximum-throughput-slice-mid} and \ref{fig:maximum-throughput-slice-high} show a detailed breakdown of where an average request spends time in the middleware.
The top plots show the absolute time spent, broken down by operation.
The bottom plots show the same data, but the time is show in percentages relative to the total amount of time spent for an average request.

Note that we see that these response time plots are very similar in shapes to the same response times as seen by the client.
Anything else would not make sense and would indicate a fault somewhere.

In the case of one thread per read pool, we can see that the majority of an average request's time is spent on the queue, waiting to be processed.
Using more threads per read pool results in the average request spending more time interacting with the server.
This is because the threads start contending for for both access to the network and for CPU time.
As the number of threads per pool increases further, the interaction time takes over most of the time spent.
This corresponds to the predictions

\begin{figure}[H]
  \centering
    \begin{subfigure}{0.48\textwidth}
    \includegraphics[width=\textwidth]{\plot{remote-maximum-throughput-slice-absolute-1-slice.png}}
    \includegraphics[width=\textwidth]{\plot{remote-maximum-throughput-slice-relative-1-slice.png}}
    \end{subfigure}
    \begin{subfigure}{0.48\textwidth}
    \includegraphics[width=\textwidth]{\plot{remote-maximum-throughput-slice-absolute-5-slice.png}}
    \includegraphics[width=\textwidth]{\plot{remote-maximum-throughput-slice-relative-5-slice.png}}
    \end{subfigure}
  \caption{Maximum Throughput}
  \label{fig:maximum-throughput-slice-low}
\end{figure}
\begin{figure}[H]
  \centering
    \begin{subfigure}{0.48\textwidth}
    \includegraphics[width=\textwidth]{\plot{remote-maximum-throughput-slice-absolute-9-slice.png}}
    \includegraphics[width=\textwidth]{\plot{remote-maximum-throughput-slice-relative-9-slice.png}}
    \end{subfigure}
    \begin{subfigure}{0.48\textwidth}
    \includegraphics[width=\textwidth]{\plot{remote-maximum-throughput-slice-absolute-13-slice.png}}
    \includegraphics[width=\textwidth]{\plot{remote-maximum-throughput-slice-relative-13-slice.png}}
    \end{subfigure}
  \caption{Maximum Throughput}
  \label{fig:maximum-throughput-slice-mid}
\end{figure}
\begin{figure}[H]
  \centering
    \begin{subfigure}{0.48\textwidth}
    \includegraphics[width=\textwidth]{\plot{remote-maximum-throughput-slice-absolute-17-slice.png}}
    \includegraphics[width=\textwidth]{\plot{remote-maximum-throughput-slice-relative-17-slice.png}}
    \end{subfigure}
    \begin{subfigure}{0.48\textwidth}
    \includegraphics[width=\textwidth]{\plot{remote-maximum-throughput-slice-absolute-21-slice.png}}
    \includegraphics[width=\textwidth]{\plot{remote-maximum-throughput-slice-relative-21-slice.png}}
    \end{subfigure}
  \caption{Maximum Throughput}
  \label{fig:maximum-throughput-slice-high}
\end{figure}

\section{Effect of Replication}

% Explore how the behavior of your system changes for a 5\%-write workload with S=3,5 and 7 server backends and the following three replication factors:
% \begin{itemize} 
% \item Write to $1$ (no replication) 
% \item Write to $\ceil{\frac{S}{2}}$ (half) 
% \item Write to all 
% \end{itemize}
% 
% Answer at least the following questions: Are \texttt{get} and \texttt{set} requests impacted the same way by different setups? If yes/no, why? Which operations become more expensive inside the middleware as the configuration changes? How does the scalability of your system compare to that of an ideal implementation? Provide the graphs and tables necessary to support your claims.

\begin{figure}[H]
  \centering
  \input{\genfile{remote-replication-effect-table.tex}}
  \caption{Replication effect experiments}
  \label{fig:replication-effect-experiment}
\end{figure}

\subsection{System under test}

The second experiment aims to explore the effect of using more replication on performance.
The clients will use a 5\%-write workload.
There will be $3$, $5$ or $7$ server backends.
Three configurations of the replication factor will be explored: No replication $(R=1)$, half replication $(\lceil \frac{S}{2} \rceil$ and full replication $(R=S)$.
The details of the system under test can be found in figure \TODO{which figure?}.


\subsection{Hypothesis}

Each request is instrumented with timestamps that, among other things, indicate the time that a first request to any server was sent, and the time that the last response was received from all servers.
The time between these two timestamps is called the 'interaction time' because that is when it is interacting with the servers.
In the case of read requests, there is always only one servert to contact, so this interaction time signifies the time it takes for the server to respond.
In the case of write requests, the interaction time signifies the time it takes for all servers to respond.
In the case of no replication that is just one server, but in the case of full replication that is all the servers.

As such, we expect the interaction time for write requests to increase as we increase the replication factor.
Moreover, we expect this effect to be more prominent the more servers are used.
For read requests, we don't expect the interaction time to increase.

\subsection{Results and analysis}

% Also make plots that put the number of servers on the x axis for each replication factor.

\begin{figure}[H]
  \centering
    \includegraphics[width=0.31\textwidth]{\plot{remote-replication-effect-replication-analysis-id-3.png}}
    \includegraphics[width=0.31\textwidth]{\plot{remote-replication-effect-replication-analysis-id-5.png}}
    \includegraphics[width=0.31\textwidth]{\plot{remote-replication-effect-replication-analysis-id-7.png}}
  \caption{Effect of replication factor}
  \label{fig:replication-analysis}
\end{figure}

\begin{figure}[H]
  \centering
    \includegraphics[width=0.31\textwidth]{\plot{remote-replication-effect-replication-analysis-rev-0.png}}
    \includegraphics[width=0.31\textwidth]{\plot{remote-replication-effect-replication-analysis-rev-1.png}}
    \includegraphics[width=0.31\textwidth]{\plot{remote-replication-effect-replication-analysis-rev-2.png}}
  \caption{Effect of replication factor}
  \label{fig:replication-analysis}
\end{figure}

\begin{figure}[H]
  \centering
    \includegraphics[width=0.48\textwidth]{\plot{remote-replication-effect-replication-cost-analysis-3-read.png}}
    \includegraphics[width=0.48\textwidth]{\plot{remote-replication-effect-replication-cost-analysis-3-write.png}}
    \includegraphics[width=0.48\textwidth]{\plot{remote-replication-effect-replication-cost-analysis-5-read.png}}
    \includegraphics[width=0.48\textwidth]{\plot{remote-replication-effect-replication-cost-analysis-5-write.png}}
    \includegraphics[width=0.48\textwidth]{\plot{remote-replication-effect-replication-cost-analysis-7-read.png}}
    \includegraphics[width=0.48\textwidth]{\plot{remote-replication-effect-replication-cost-analysis-7-write.png}}
  \caption{Effect of replication factor}
  \label{fig:replication-analysis}
\end{figure}

\section{Effect of Writes}

% In this section, you should study the changes in throughput and response time of your system as the percentage of write operations increases. Use a combination of 3 to 7 servers and vary the number of writes between 1\% and 10\% (e.g. 1\%, 5\% and 10\%). The experiments need to be carried out for the replication factors R=1 and R=all.  
% 
% For what number of servers do you see the biggest impact (relative to base case) on performance? Investigate the main reason for the reduced performance and provide a detailed explanation of the behavior of the system. Provide the graphs and tables necessary to support your claims.
 
\begin{figure}[H]
  \centering
  \input{\genfile{remote-write-effect-table.tex}}
  \caption{Write effect experiments}
  \label{fig:write-effect-experiment}
\end{figure}

\subsection{System under test}

In this last experiment, we investigate the results of increasing the percentage of the workload consists of write requests on performance.
Between 3 and 7 servers will be used with either no replication or full replication.
The percentage of the requests that consists of writes will be varied between 1\% and 10\%.
We call this percentage the write percentage.

\subsection{Hypothesis}

As investigated in the previous section, writes are much more expensive in a setup with full replication than in a setup with no replication.
This means that we can expect there to be a difference in performance between the no replication situation and the full replication situation for any given non-zero write percentage.
We expect that the system with full replication will perform worse and that this difference will increase as the write percentage increases.
Moreover, we expect this difference to be greater if more servers are used.
Concretely we expect a lower throughput and higher response times in the case of full replication relative to the case of no replication.
We also expect this difference to be larger for a setup with more servers and/or a greater write percentage.


\subsection{Results and analysis}

\begin{figure}[H]
  \centering
    \includegraphics[width=0.48\textwidth]{\plot{remote-write-effect-write-analysis-tps-3.png}}
    \includegraphics[width=0.48\textwidth]{\plot{remote-write-effect-write-analysis-tps-5.png}}
    \includegraphics[width=0.48\textwidth]{\plot{remote-write-effect-write-analysis-tps-7.png}}
  \caption{Effect of write percentage: Throughput}
  \label{fig:write-analysis-tps}
\end{figure}
\begin{figure}[H]
  \centering
    \includegraphics[width=0.48\textwidth]{\plot{remote-write-effect-write-analysis-resp-3.png}}
    \includegraphics[width=0.48\textwidth]{\plot{remote-write-effect-write-analysis-resp-5.png}}
    \includegraphics[width=0.48\textwidth]{\plot{remote-write-effect-write-analysis-resp-7.png}}
  \caption{Effect of write percentage: Response time}
  \label{fig:write-analysis-resp}
\end{figure}

\newpage

\section{Logfile listing}

\input{\genfile{report-2-logfile-listings.tex}}
 
\end{document}
